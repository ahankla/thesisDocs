\chapter{Introduction}
Black holes are now a popular fixture in science fiction books and movies. From tiny primordial black holes (XXXXXX) to 

but things rarely fall radially inward. Usually the matter has some angular momentum. Since angular momentum is conserved, a disk usually forms around the black hole as a repository for rotating matter. Processes within the disk then lead to turbulence and an effective friction that results in an outward transport of angular momentum and hence the falling in, or ``accretion'', of matter. For reasons discussed in Chapter~\ref{sec:AFs}, the infalling matter (mostly hydrogen) heats up past the ionization threshold. We shall thus refer to the matter as a plasma; different types of plasma and the effect on accretion flows will be discussed in Chapter~\ref{sec:mhd}. For now, we motivate this paper by noting that gravitational binding energy is released as matter falls into a black hole, a process that is among the most efficient energy sources in the universe (several hundred times more efficient than fusion~\cite{Blandford1999}). This energy presumably goes into radiation and thus 

These processes and the debate surrounding them are discusssed throughout the paper, particularly 

accretion=falling in. But how does accretion work?

Yay introduction.
\section{AGN and Accretion Disk Observations}
How explain low luminosity from AGN (Sgr A*), x-ray binaries in quiescence
*rotation measure  stability in Sgr A* (Sharma, Quataert, Stone 2008)
 
