\chapter{Introduction}\label{chap:introduction}
XX add pretty picture?
XX protoplanetary disks?
XX quote Broderick2011 that ``ab initio calculations are beyond our present capability''

\section{Overview}
The discipline of plasma physics encompasses a vast variety of plasmas, stretching from earthly laboratory-made fusion plasmas to astrophysical plasmas such as the intracluster medium and accretion flows onto compact objects. What separates this discipline into different subfields is the immense variety of scales in terms of distance, density, and temperature. Although the theory of magnetohydrodynamics is scale-less, the different relationship between parameters means that the plasma inside a tokamak is not usefully described by the same set of equations as a disk of matter around the black hole at the center of the galaxy, where the distances are tens of orders of magnitude greater, densities are tens of orders of magnitude times smaller, and temperatures ten times lower (see Table~\ref{table:paramCompare}).\\
\\
All plasmas can in principle be described by a collection of equations describing the Lorentz force and other interactions on every single particle in the plasma. As can easily be imagined, however, the task of following billions of particles is intractable both analytically and computationally. Different sets of assumptions allow the impossible equations to be reduced to something useful in their respective situations. For instance, plasmas in which the particles do not collide often if at all (termed ``weakly collisional'' or ``collisionless'', respectively), need to be evolved using a distribution function that takes into account the spread of particle velocities: a kinetic theory. In contrast, when the particles in a plasma collide many many times before they travel any meaningful distance in a system, the plasma can be treated as a fluid. Said fluid has only one, ``bulk'', velocity at any given point. The fluid mechanical approach is a further simplification of the full kinetic theory. As such, we would not expect a fluid treatment of collisionless plasmas to hold much weight.\\
\\
It is perhaps surprising then, that it is common practice to do so. This unjustified assumption is not so hard to understand given the conceptual and practical simplification that the fluids model presents: instead of evolving six degrees of freedom, there are only three. It is also more intuitive to think about fluids as we have every day experience with them. So is this assumption valid? To date an investigation of the validity of this assumption seems to be absent from the astrophysical literature, although there might be such works among the fusion community.\\
\\
The main goal of this thesis is to explore the viability of modelling a collisionless plasma with a modified fluid closure. If such an approximation is found to exist, then the assumptions of the past studies listed above are validated and the works stand on firmer theoretical ground. The approximation would also path the way for further studies of collisionless plasmas, easing not only the conceptual difficulty of a full kinetic theory but also the computational nuisance of particle-in-cell (PIC) simulations, which are limited by the extreme amount of resources they consume (a PIC simulation that takes millions of cpu-hours can be done in tens of cpu-hours using a fluid model).

\section{Collisionless Plasmas}
More formally, a collisionless plasma is one for which the particles on average travel longer than the scales one is interested in without colliding. This means that the length scales of interest (the radius of an accretion disk, for example, or the distance between the sun and the earth for the solar wind) $L$ is much less than the mean free path $\lambda_{mfp}$ of particles. In these cases the magnetic field is also strong enough such that the Larmor radius $\rho$ is much less than both length scales of interest and the mean free path (a so-called ``magnetized'' plasma). We therefore have the ordering
\begin{equation}\label{eq:collLessOrdering}
  \rho\ll\ L \lesssim \lambda_{mfp}
\end{equation}
For weakly collisional plasmas, the mean free path is on the order of the length scales of interest. These orderings are opposed to fluid models in which both the mean free path and the Larmor radius are much less than the length scales of interest ($\lambda_{mfp}\ll L$, $\rho\ll L$). \\
\\
Although the fluid description of a plasma (called ``magnetohydrodynamics'', MHD) has yielded a useful framework for fusion plasmas and the interstellar medium and connects nicely to the entire field of fluid dynamics (hydrodynamics---magnetohydrodynamics with no magnetic fields), the range of systems underneath the ``collisionless plasma'' umbrella is still quite large and therefore worth studying. Collisionless plasmas are most typically found in 1) the intracluster medium between galaxies, 2) radiatively inefficient accretion flows (RIAFs) around black holes, and 3) the solar wind~\cite{Kunz2010}. Table~\ref{table:paramCompare} shows these collisionless plasma parameters in comparison to collisional plasmas such as magnetically-confined fusion plasmas. Here we often use the term ``collisionless'' also to describe weakly collisional systems such as RIAFs. \\
\\
Collisionless plasma can be created and studied on earth in a laboratory by scaling parameters and using high-power lasers~\cite{Courtois2004,Romagnani2008,Dean1971}. For instance, one experiment uses a density of $10^{18}~cm^{-3}$ with a mean free path of $1~mm$ and a magnetic field of $7.4e4$ G to get the same ordering between parameters~\cite{Courtois2004}. A brief overview of the three aforementioned types of weakly collisional/collisionless plasma follows.\\
\\
The solar wind, a stream of particles released from the sun, is a weakly collisional plasma that has been regularly studied thanks to the Voyager missions and the Solar Wind Ion Analyzer aboard the MAVEN (Mars Atmosphere and Volatile Evolution Mission) \cite{NASA2017}. Another solar wind experiment, Helios 2, observed anisotropic electron distribution and resulting anisotropic heat flux~\cite{Pilipp1987}. In analyzing the data, many papers acknowledge the inability of the MHD fluid model to capture kinetic effects. This assumption is often justified by likening the results of applying the model to hydrodynamic turbulence and noting success~\cite{Matthaeus1982,Goldstein1995} or claiming the inability to analyze kinetic instabilities theoretically~\cite{Schwartz1980,Pudovkin1985}. Others just accept the common standard and do not mention the fluid assumption~\cite{Gosling1996}. \\
\\
The IntraCluster Medium (ICM), the gas that inhabits the space between galaxies in a galaxy cluster, is super hot and thus radiates in the X-ray regime. Again, papers do not justify the fluid assumption or simply do not attempt to explain observations in the context of a broader model~\cite{Fabian1994,Carilli2002,Mendygral2012}. Recent investigations on the MagnetoThermal Instability (MTI) and Heat-flux-driven Buoyancy Instability (HBI) in the ICM have treated the ICM as a modified fluid~\cite{Chandran2006,Parrish2008a,Parrish2008b,Parrish2007,Parrish2005}.\\
\\
Radiatively-inefficient accretion flows (RIAFs) have also often been treated as a fluid, usually with qualifications about such ad hoc assumptions~\cite{Dexter2013,Hawley2001,Stone1996,Jiang2013,Jiang2014,Stone1994,Turner2002,Sano2004}. RIAFs are often found in a binary system, especially an x-ray binary system where a black hole is accreting matter off a white dwarf or other object. It is hoped that treating RIAFs correctly can explain observations of x-ray binary outbursts, or transitions from the dormant quiescent soft state to the active hard state and vice versa~\cite{McClintock2006,Das2013,Das2013b,Niedzwiecki2014,Sadowski2016}. Another possible explanation is black hole jets~\cite{Veledina2013,Fender2009,Nixon2014}, which are very complex phenomena (see~\cite{Hawley2015} for a review), so we will stay away from x-ray binaries in this thesis.\\
\\
Work involving accretion flows is also relevant to the supermassive black hole at the center of our galaxy, Sagittarius A*, as well as other, less well-resolved supermassive black holes like M87~\cite{Ressler2015,Oezel2001,Foucart2015,Broderick2015} and GX 339-4~\cite{Plant2014}. The accretion disk around Sagittarius A* is many times dimmer than one might expect, knowing that the gravitational energy of in-falling matter must go somewhere---and where does it go, if not into radiation that can then be detected on Earth? Current models suggest that the accretion disk is heated up, resulting in a hot flow whose mean free path between particles is correspondingly large: a collisionless plasma. Such flows are termed ``radiatively inefficient accretion flows'' (RIAFs) and are thought to effectively model low-luminosity active galactic nuclei (LLAGN) such as Sagittarius A*~\cite{Rohan1998,Broderick2011,Broderick2009,Dexter2013,Yuan2003}. \\
\\
This thesis will concentrate on collisionless plasmas in the context of RIAFs. This choice is motivated both by a recent paper that presents the first kinetic simulation of how accretion in a RIAF happens locally~\cite{Kunz2016} and by hints at the inability of a fluid model to capture the correct growth rates of kinetic phenomena~\cite{Sharma2004}. Although there has been some study on modified fluid closures to capture kinetic physics~\cite{Sharma2006,Sharma2007,Sharma2003,SharmaThesis}, these studies use a different formalism. Now that a 3D kinetic simulation has been performed by~\citet{Kunz2016}, a direct comparison between a modified fluid closure and kinetic theory is possible. The paper uses methods that are easily replicated using the Athena code developed by~\citet{Stone2008}, in particular the same local shearing-box method~\cite{Stone2010}. A comparison is thus easily facilitated and meaningful. \\
\\
This paper is complementary to other current research in the field of accretion disk physics that often assumes a high degree of collisionality. For example, there is a push to include general relativity in accretion disk calculations in order to describe observations~\cite{Moscibrodzka2014,Ressler2015,Shiokawa2013,Sadowski2016,Niedzwiecki2014,Narayan1998}. This paper addresses the fundamental assumptions of these papers and assesses their validity, providing the groundwork for extending magnetohydrodynamics to relativistic systems and encompassing other effects important for reconciling observations and current models.\\
\\
A fluid closure to kinetic physics has far-reaching consequences. If proper parameters are found that sufficiently imitate the kinetic physics of RIAFs, then this model can be extended to global simulations and enables the exploration of the parameter space of collisionless plasmas. Such exploration is currently prohibitively expensive computationally as mentioned previously because PIC simulations are required. If a fluid model is achieved, then the simulations are much more manageable, allowing for more thorough scans of, for example, magnetic field strength. A fluid model closure to kinetic physics is also interesting in a conceptual sense since it would mean that the six phase-space degrees of freedom could be reduced to only three position-space degrees of freedom. \\
\\
The structure of this paper is as follows: the necessary background to understand the fundamental plasma physics question is explained in Chapter~\ref{chap:plasmaphysics}, which covers both important parameters such as the mean free path as well as the different models and closures that appear in various regimes, such as kinetic theory and magnetohydrodynamics. The next chapter zooms in to the astrophysical context of black hole accretion disks, motivating this paper's consideration of low luminosity active galactic nuclei through observational evidence and outlining basic accretion disk mechanics (Chapter~\ref{chap:astrophysics}). The actual physical mechanism for accretion, a linear MHD instability called the magnetorotational instability (MRI), is outlined in Chapter~\ref{chap:compMRI} using both analytic theory and the main tool of the rest of this thesis---simulations. By building from the simplest MHD theory (ideal MHD) to more complex, resistive MHD, principles of numerics such as choosing a proper box size and resolution are reviewed, allowing a smooth transition into the original research of this thesis found in Chapter~\ref{chap:kinbrag}. Chapter~\ref{chap:kinbrag} uses the anisotropic viscosity along magnetic field lines as explained in Chapter~\ref{sec:bragMHD} in the context explained in Chapter~\ref{chap:astrophysics} in an attempt to capture kinetic physics in a modified fluid closure. 

\begin{table}[h]
  \begin{tabular}{c|c|c|c|c|c|c|c}  
    & L (cm) & n (cm$^{-3}$) & T (eV) & B (G) & $\rho$ (cm) & $\lambda_{mfp}$ (cm) & Description \\ \hline\hline
    ICM & 6.2e23 & 5.0e-3 & 8.0e3 & 1.0e-6 & 1.3e10 & 9.5e21 & KINETIC \\ 
    RIAFs & 1e13 & 1.0e2 & 2.0e3 & 1.0e-3 & 1e6 & 1e17 & KINETIC \\ 
    Solar wind & 1.5e13 & 1.0e1 & 1.0e1 & 1.0e-4 & 4.6e6 & 1.2e13 & KINETIC  \\ \hline
    ISM & 3.1e20 & 1.0e0 & 1.0e0 & 5.0e-6 & 2.9e7 & 1.3e12 & FLUID \\ 
    JET & 1.0e2 & 1.0e14 & 1.0e4 & 3.0e4 & 4.8e-1 & 1.4e6 & FLUID  \\ 
  \end{tabular}
  \caption{Comparison of parameters of different plasmas found in space and on earth. ICM: intracluster medium. RIAF: radiatively-inefficient accretion flow. ISM: interstellar medium. JET: Joint European Torus (a tokamak). From~\cite{Kunz2010,AST521Pset1}.}
  \label{table:paramCompare}
\end{table}

\section{Codes: Athena4.2 and Pegasus} \label{sec:codes}
The systems under consideration in this thesis are extremely complicated and thus require the use of simulations to model on large time or length scales. Two codes are used for this purpose: one, an MHD solver, the other, a hybrid-kinetic particle-in-cell (PIC) code.\\
\\
The code in use in Chapters~\ref{chap:compMRI} and~\ref{chap:kinbrag} to simulate MHD systems is Athena4.2 (henceforth referred to as Athena). Athena, a response to the older code ZEUS, uses a higher-order Godunov scheme for flexibility and the constrained transport technique to ensure a divergence-free magnetic field. It is a highly-modularized grid-based code with additions such as adaptive mesh refinement (AMR) capabilities, special relativity, and dust~\cite{Stone2008,Stone2009}. The shearing box approximation (as explained in Section~\ref{ssec:shearingbox}) has also been implemented~\cite{Stone2010}. A new version of Athena, Athena++, more easily integrates general relativity and allows for better Riemannien solvers~\cite{White2016Thesis,White2016}.\\
\\
For collisionless plasmas, the distribution function itself must be evolved. Such evolution is accomplished with a so-called ``particle-in-cell'' or PIC code. Because a fully-kinetic code usually requires compromising assumptions such as reduced speed of light or a smaller ion-to-electron mass ratio, hybrid-kinetic codes such as PEGASUS, the one used in~\cite{Kunz2016} that this thesis compares its results to in Chapter~\ref{chap:kinbrag}, are perhaps more useful. PEGASUS itself treats electrons as a massless fluid, while ions are treated kinetically. This assumption is valid since ions are much hotter than electrons since electrons can radiate efficiently~\cite{Das2013}. PEGASUS is a second-order accurate code that uses a three-stage predictor-predictor-corrector algorithm for integration. It also uses the constrained transport method as Athena does to enforce a divergence-less magnetic field and implements the shearing box method~\cite{Kunz2014}. 

