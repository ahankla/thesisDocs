\chapter{History and Simulation of Accretion Disks}
This chapter examines the development of the theory behind accretion disks, with the ultimate goal of framing the past and current research on accretion disks. Due to the complexity of the equations involved (see Chapter~\ref{ch:plasmaTheory}), numerical simulations play an integral role in understanding accretion flows. However, said complexity also means that the power of numerical simulations to resolve interesting features has been and is limited by the computational ability and algorithms utilized. The theory and numerical simulations of accretion disks have thus evolved in step, with new insights coming from both sides. As such, it is difficult to disentangle the pure theory from the numerical simulations and the following chapter does not attempt to do so. It is instead an intertwined, mostly chronological account that showcases the thought processes and impetus behind new developments.\\
\\
We first trace back to some of the early ideas of accretion disks and how they were formed (Section~\ref{sec:early}), leading into Section~\ref{sec:sources}'s discussion of possible sources of turbulence and then Section~\ref{sec:current}'s modern hypotheses and what this means for the original research of this paper (e.g. Chapter~\ref{ch:duhRESEARCH}). 


\section{Early Thoughts on Accretion}\label{sec:early}
The study of accretion in the modern era emerged from simpler questions than that of AGN luminosity. In 1941, Kuiper~\cite{Kuiper1941} recognized the formation of a disk during mass transfer in a binary-star system, while Hoyle and Lyttleton~\cite{HoyleLyttleton1939} attributed the climatic variation on Earth to the accretion of interstellar matter around the sun. Once compact x-ray sources such as AGNs were actually observed, the role of disks in the production of radiation was explored more thoroughly~\cite{PrendergastBurbidge1968}. At first, however, the basic physics of accretion disks was under discussion.\\
\\
As mentioned in Section~\ref{sec:XXXXXXXXXXXXXX}, accretion involves the outward transport of angular momentum, which means the slowing down of particles as they drop into closer orbits (Section~\ref{sec:XXmechanics?XX}). It seems natural to explain this slowing down via friction; in an accretion disk (or torus), the matter at different radii are not moving at the same velocity (i.e. there is a shear) and hence one might think that there is a sort of coefficient of kinetic friction between particles that slows down their movement and causes them to accrete. The idea that this ``molecular'' or ``shear'' viscosity could explain accretion rates was tempting, but not supported by simulations. Early simulations solving the MHD equations for a disk showed accretion rates on the order of XXXXXXXXXX (CITE: XXXXXXXSTILL need some simulations showing molecular rates!!XXXXXXXXXxx), XXXXXXxalso observations!XXXXXXx; however, according to the standard values of molecular viscosity (in, for example,~\cite{Spitzer1962}XXXXXXXX make sure cite format right), the standard values of molecular viscosity are around XXXXXXXx. This fantastic difference between theory and simulations resulted in several new ideas for explaining the transport of angular momentum and led to the formulation of one of the most well-known models for thin disks--the $\alpha$-disk.\\
\\
The seminal paper of ~\cite{SS1973} Xmake sure format right!!!XXXXXx explores accretion disks in the context of a binary star system. It essentially characterizes ignorance in the accretion rate via the parameter $\alpha$, defining the tangential stress $w_{r\phi}=\alpha\rho v_s^2$, where $v_s$ is the sound speed such that $\rho v_s^2/2$ is the disk matter's thermal energy density, although definitions vary to order unity across sources~\cite{SS1973}. This formulation temporarily removed the need to explain the source of the viscosity and provides a parameter that is easy to tweak in numerical simulations. Although the original paper takes $\alpha$ as a constant for simplicity, it is generally a function of radius. The relevance of the $\alpha$ parameter is apparent even today, as it is a simple way to gain intuition in accretion problems despite its debated value~\cite{PennaEA2013}. \\
\\
Despite its intuitive usefulness, the $\alpha$ prescription offers no mechanism for the transport of angular momentum. XXXXXXXXX proposed that, while pure molecular viscosity could not explain the observed accretion rates, an ``effective'' viscosity due to eddy interaction could do the job~\cite{XXXXXX}. In other words, turbulence would generate eddies whose interactions would manifest similar to a viscosity. The problem became to find the source of the turbulence that would lead to outward angular momentum transport. This question is the subject of the next section.

\section{Potential Sources of Turbulence}
Supposing that an effective viscosity generated by turbulence can explain observed and simulated accretion rates, the question becomes: what causes this turbulence? 
\subsection{Large Reynolds Number}
Some, influenced by laboratory fluid mechanics, believed that the sheer property of having a high Reynolds number (the product of a characteristic velocity and length scale divided by the viscosity; huge in astrophysical flows due to the large length scales involved) satisfactorily accounted for the needed turbulence. The mechanism at hand is called ``vortex stretching'': due to vortex conservation, the stretching of vortices in a shear flow increases the circulation velocity around the vortex tube. This allows for free energy to be extracted from the shear flow~\cite{BH1998}.\\
\\
Others, however, suspected that, at least in accretion disks, the flow was fundamentally different than those shear flows explored in the aforementioned fluid mechanics labs. Indeed, as demonstrated nicely in~\cite{BH1998:SectionIII.c.1} and reproduced in Section~\ref{sec:MECHANICSkeplerianvsshearflow}, Keplerian flows are stable against perturbations (i.e. experience no turbulence) where shear flows are not (given the Rayleight stability criterion is satisfied). The difference is due to epicycles in Keplerian flows, which sink the energy that would otherwise devolve into prominent disturbances. A high Reynolds number is not enough to explain the necessary turbulence.
\subsection{Convective and Self-gravitating Instabilities}
It was long thought that hydrodynamic instabilities could lead to turbulence in accretion disks. One such idea proposed by~\cite{Paczynski1978}XXXXformatXxx in very flat disks was that gravitational instabilities similar to the Jeans instability could lead to internal heating which would in turn limit the development of the instability, maintaining a somewhat unstable state. While the paper raised important questions concerning heat transport, the effect of the instability is small in hot disks~\cite{Spruit2009}.\\
\\
Convective instabilities have garnered the most interest in terms of generating hydrodynamic turbulence. After all, the Schwarzschild condition...

\subsection{Nonlocal Effects}
If the mechanism for producing turbulence were global, its effect would not be captured by a local viscosity parameter.~\cite{Spruit2009} mentions several possibilities for generating turbulence this way, including waves and shocks created by tidal forces. These effects can produce accretion at rates up to $\alpha=.01$, but only in hot disks. Global disk winds, of the type suggested by~\cite{Blandford1976}XXformat?XX, could also transport angular momentum. These magnetically-driven winds could theoretically sweep matter around in such a way as to account for the high accretion rates without a viscosity while also helping account for AGN jets~\cite{Koenigl1989}; however, the presence of these winds in all accretion disks is debated. A more universal and fundamental explanation seems more likely.

\subsection{Magnetic Fields}
Magnetic fields were thought to serve an amplifying role in turbulence transport. That is, with pre-existing turbulence, magnetic fields would tangle and speed along the transportation of magnetic fields~\cite{ShakuraSunyaev1973}. It was thought that the magnetic pressure and pressure due to turbulence were distinct, and that magnetic pressure would be insignificant in disk situations~\cite{LyndenBell1969}, or would require large magnetic fields on the order of $10^7-10^8$~G to balance the gravitational pressure of infalling gas~\cite{TrevesAl1988}. The magnetic field was mainly considered to be important due to consequences of cyclotron radiation as a cooling mechanism~\cite{Shapiro1973}. It was not until the early 1990s that the full significance of magnetic fields was appreciated.\\
\\
In 1991, Balbus and Hawley~\cite{BalbusHawley1991a,BalbusHawley1991b} closed the conceptual circle by showing that turbulence resulted directly from a weak magnetic field. Pre-existing turbulence was not needed; the entire sequence of generating turbulence and transporting turbulence and angular momentum could be derived as a result of a linear instability in the MHD equations (see Section~\ref{ssec:mriDeriv}). Numerous numerical simulations have since confirmed the important role of magnetic fields in accretion processes. The next section will give an overview of these numerical simulations and thereby place the simulations in section~\ref{sec:duhRESEARCH} in context. 

\section{Post-MRI Discovery Developments}

shearing box
artificial viscosity
numerical viscosity/resistivity


why reproduce stone pringle2001...



**lead into accretion theory? then MRI?

\subsection{Current Research}
Although the basic source of turbulence in accretion disks seems to be the MRI, there are still many alterations to
GR simulations Foucart et al 2016
Still 2D!
Including 2-temperature model, as discussed inXXx. Sadowski et al 2016
Outflows and jets (Bu, Wu, Yuan 2016)
still no 3d mti rotating? parrish  stone=nonrotating?
KINETIC!


