\chapter{Ideal MHD: Theory and Simulations}
%-----------outline--------------%
%intro to mhd equations
%intro to mri
%simulation intro
%mri sims
%---------------------------------%
Understanding magnetohydrodynamics (MHD) is a topic worthy of a semester-long graduate class at least. This chapter briefly presents some basic plasma physics and physically motivates the driving equations of magnetohydrodynamics, which are fluid equations . The more rigorous derivation involves taking moments of the distribution function that will be introduced in Chapter~\ref{chap:KINETICS}.\\
\\
The ultimate goal of this thesis, however, is to study a particular phenomenon (the Magneto-Rotational Instability--MRI) in a particular regime (anisotropic viscosity) of non-ideal MHD. As such, for the sake of compactness we only summarize a portion of the necessary theory. This chapter explores ideal MHD (Section~\ref{sec:idealXXX}) and introduces the physics of the MRI in this most basic setting and how they manifest in simulations (Sections~\ref{sec:MRItheoryXXX} and~\ref{sec:MRISimsXXXX}, respectively). Its results are extended to include viscosity and resistivity in Chapter~\ref{chap:nonIdealMHDXXXXXX}, and later in Chapter~\ref{chap:kinBragXXX}, where the validity of the MHD approximations are evaluated via a comparison to the kinetic regime (also outlined in Chapter~\ref{chap:kinBragXXX}). First, however, some context in general plasma physics is in order. 

\section{Properties of a Plasma}
Plasma physics applies to a wide range of subject areas, from magnetic-confinement fusion pursuits like the tokamak ITER~\cite{X} and the stellarator Wendelstein 7-X~\cite{XXX} to a variety of astrophysical situations, including the sun's corona and protoplanetary disks. The uniting theme across these different disciplines is the plasma: so what exactly is a plasma? \\
\\
A plasma is the so-called ``fourth state of matter'', coming after the gas phase in the increasing kinetic energy hierarchy solid-liquid-gas: that is, the inetic energy of a plasma particle is much greater than its potential energy. A plasma is made up of neutrals and the result of the neutrals' ionization: that is, ions and electrons. The basic physics of single-particle motion in electric and magnetic fields (for example, $\nabla B$ drift and $E\times B$ drift) apply to every single particle. Given the enormous quantity of particles (as defined by the plasma parameter $\Lambda$ in Section~\ref{ssec:debye}, upwards of XXXXXXXXXX particles per cubed meter), working analytically or simulating such a situation for each individual particle is near impossible. Indeed, this is why the kinetic theory is so complicated and requires particle-in-cell simulations (Chapter~\ref{chap:kinBragXXX}, Section~\ref{sec:PICXXXXXX}). The task of the Section~\ref{sec:idealXXX} is to make this feasible via fluid equations. 

\subsection{Debye Length and Shielding} \label{ssec:debye}
Since the ions and electrons in a plasma have opposite charge, they tend to attract, leading to the phenomenon of Debye shielding. Electrons of negative charge tend to cluster around a positive test charge and effectively shield the potential such that it is no longer Couloumbic ($V\propto1/r^2$) but rather exponentially decays ($V\propto e^{-r/\lambda_D}/r$). The distance over which the potential is screened is the total Debye length, which is related to the electron and ion Debye lengths $\lambda_e$ and $\lambda_i$ by:
\begin{equation}
  \lambda_D^{-2}=\lambda_e^{-2}+\lambda_i^{-2}
\end{equation}
where each species Debye length $\lambda_j$ is given by
\begin{equation}
  \lambda_j=\sqrt{\frac{T_j}{4\pi n_0 e^2}}
\end{equation}
where $T_j$ is the species equilibrium temperature in units of energy (Boltzmann's constant $k_b=1$), $n_0$ is the density of each species far away from the test charge, and $e$ is the charge on an electron. The exponential decay comes from the Boltzmann distribution since the system is at equilibrium. Detailed derivations can be found in any standard plasma physics introduction~\cite{Nicholson, GurnettBhatt}. \\
\\
Going back to our definition of a plasma, requiring that the potential energy of a particle is much less than its kinetic energy leads to the fact that the plasma parameter $\Lambda_s\equiv n_0\lambda_s^3$ of each species is
\begin{equation}
  \Lambda_s\gg1
\end{equation}
This means that, in a plasma, there are many particles in a cube of the Debye length (alternative definitions define the plasma parameter as the number of particles within a sphere of radius the Debye length). \\
\\
Other important quantities include the mean free path $\lambda_{mfp}$ of a plasma particle, that is, how far it travels on average before it collides with another particle. This is related to the thermal velocity $v_T$ and collision frequency $\nu$ by $\lambda_{mfp}=v_T/\nu$. \\
\\
We can also define the gyrofrequency of a species $\Omega_s$ by looking at the equation of motion of a particle in a magnetic field, resulting in $\Omega_s=q_sB/m_sc$. Note that this is half of the Larmor frequency (XXXXXXXXXXX). From the gyrofrequency we define the mean gyroradius $\rho_s$ as the radius of the circle of a particle traveling at the thermal velocity: $\rho_s\equiv v_{T_s}/\Omega_s$. This quantity in particular will prove important in later chapters (i.e. Chapter~\ref{chap:kinBragXXXXX}), where ordering such that the gyroradius is much less than the mean free path will lead to anisotropic viscosity along the magnetic field lines.\\
\\
With the basic quantities of $\lambda_{mfp}$ and $\rho_s$ in mind, we can now turn to the assumptions behind ideal MHD.

\section{Ideal MHD Equations}
Ideal MHD treats the different species of electrons and ions as fluids. In the broader MHD context, each species is at a different temperature, has a different gyroradius, and so on. However, single-fluid MHD, where each species has the same temperature, is still often a useful framework. In these cases the density adds linearly and other quantities are redefined as such. XXXXxx
rho and mfp\\
\\
Magnetohydrodynamics is a fluid formalism; this means that the individual particles' positions and velocities can be averaged out and the important quantities are bulk variables, like the mean flow of the fluid, density, and pressure. Treating each individual particle with a velocity distribution is characteristic of kinetic theory and the distribution function, as discussed in Chapter~\ref{chap:kinBragXXX}. We are considering a continuous source rather than discrete particles that would manifest as delta functions. \\
\\
We now proceed to motivate the equations of ideal MHD by a combination of conservation arguments and Maxwell's equations. We consider for now a single species. First note that the standard continuity equation, saying that the fluid elements cannot be created or destroyed, leads to the first equation:
\begin{equation}
  \frac{\partial\rho}{\partial t}+\nabla\cdot\left(\rho\vec V\right)=0\label{eq:continuity}
\end{equation}
The change in the density $\rho$ at any given point is due to the fluid flowing into or out of that region, with velocity $\vec V$. \\
\\
We achieve a similar momentum conservation equation by considering a force balance: $\rho \dot V(\vec x,t)=F(\vec x,t)$, where $\rho$ is the mass density. $V$ is the velocity of a fluid element at position $\vec x$ at time $t$; $F$ is the net force per unit volume acting on the fluid element at position $\vec x$ at time $t$.\\
  \\
  It is important to note that $\dot V$ is not simply the partial derivative of the velocity with respect to time; the fluid velocity is not only changing in time but also spatially. The chain rule then gives
  \begin{align}
    \dot \vec V(\vec x,t)&=\frac{\partial \vec V}{\partial t}+\frac{dx_i}{dt}\frac{\partial \vec V}{dx_i}\\
    &=\frac{\partial\vec V}{\partial t}+\left(\vec V\cdot\nabla\right)\vec V
  \end{align}
  This total derivative is often represented as
  \begin{equation}
    \frac{D}{Dt}=\frac\partial{\partial t}+\vec V\cdot\nabla
  \end{equation}
  where $\vec V$ is the background flow of the fluid.
\\
  The forces acting on the system will be the Lorentz force law $\vec J\times\vec B/c$ and the pressure gradient force $-\nabla P$, leading to a force per unit volume of:
\begin{equation}
  F(\vec x,t)=\frac1c\vec J(\vec x,t)\times B(\vec x,t)-\nabla P(\vec x,t)
\end{equation}
Altogether then, we have our equation of momentum conservation:
\begin{equation}
  \rho\frac{D\vec V}{Dt}=\frac1c\vec J\times\vec B-\nabla P \label{eq:momcons1}
\end{equation}
Here, $\vec J$ comes from Maxwell's equation:
\begin{equation}
  \nabla\times\vec B=\frac{4\pi}c\vec J+\frac1c\frac{\partial\vec E}{\partial t}
\end{equation}
However, we can neglect the displacement current since we are here concerned with non-relativistic MHD and the displacement current's electric field will be much smaller than the magnetic field.\\
\\
The current can thus be written as
\begin{equation}
  \vec J=\frac{c}{4\pi}\nabla\times\vec B \label{eq:current}
\end{equation}
Using Eq.~\ref{eq:nrl11}, we can re-write the first term on the right hand side of Eq.~\ref{eq:momcons1} as
\begin{align}
  \frac1{4\pi}\left((\nabla\times\vec B)\times\vec B\right)&=\frac1{4\pi}\left((\vec B\cdot)\nabla\vec B+(\nabla\vec B)\cdot\vec B\right)\\
  &=\frac1{4\pi}\left(\frac12\nabla B^2+(\vec B\cdot\nabla)\vec B\right)
\end{align}
where $B^2=\vec B\cdot\vec B$. These two terms can be interpreted as magnetic pressure and magnetic tension: magnetic pressure, which pushes field lines apart, is $\frac1{8\pi}\nabla B^2$ and the magnetic tension, which tries to uncurl magnetic field lines, is $(\frac{\vec B}{4\pi}\cdot\nabla)\vec B$. It is common to see the momentum equation in the form
\begin{equation}
  \rho\frac{D\vec V}{Dt}=-\nabla\left(P+\frac{B^2}{8\pi}\right)=\left(\frac{\vec B}{4\pi}\cdot\nabla\right)\vec B
\end{equation}

We can put the continuity equation~\ref{eq:continuity} in a similar (conservative) form:
\begin{align}
  \frac{\partial\rho}{\partial t}&=-\nabla\cdot\left(\rho\vec V\right)\\
  &=\frac{\partial\rho}{\partial t}+\vec V\cdot\nabla\rho-\vec V\cdot\nabla\rho\\
  &=\frac{D\rho}{Dt}-\vec V\cdot\nabla\rho\\
  &=\frac{D\rho}{Dt}-\nabla\left(\rho\vec V\right)+\rho\nabla\cdot\vec V
\end{align}
whereby we have
\begin{equation}
  \frac{D\rho}{Dt}+\rho\nabla\cdot\vec V=0
\end{equation}
For the last ideal MHD equation, we consider Ohm's law:
\begin{equation}
  0=\vec E+\frac1c\vec v\times\vec B)
\end{equation}
XXXXxwhy
Using Faraday's law, we have
\begin{align}
  \frac{\partial\vec B}{\partial t}&=-c\nabla\times\vec E\\
  &=\nabla\left(\vec v\times\vec B\right)
\end{align}

energy/entropy equation? 

\section{The MRI in Ideal MHD: Theory}
As discussed in Chapter~\ref{chap:history}, the source of turbulence in accretion disks remained a puzzle until linear analysis by Balbus and Hawley in 1991~\cite{BH1991}. This section outlines the derivation of the MRI: its dispersion relation, maximally growing mode, and other important features. The method of choice in this paper to simulate the MRI will be discussed in the next section (Section~\ref{sec:mrisbideal}), with the actual simulation results and their interpretation coming in the section after that (Section~\ref{sec:mrisimideal}).

\subsection{MRI Spring Analogy}

\section{Simulating the MRI: the Shearing-Box Method} \label{sec:mrisbideal}
In simulating accretion disks, it is much more computationally expensive to describe the disk as a whole, including pressure gradients and the overall global structure. A smaller-scale, local simulation is also effective at demonstrating the physics involved. As noted in Chapter~\ref{sec:introXXXXX}, the series of simulations in this chapter and the next zoom in to a typical region of the accretion disk and consider the MHD equations in that blown-up portion. We now explore exactly how such local simulations work and the relevant quantities.\\
\\
The so-called ``shearing-box'' approximation has been in use since XXXXXX in studies examining the solar corona XXXXXXX etc. It was also used by Balbus and Hawley in their seminal 1991 papers showing the existence of the MRI instability~\cite{BH1991a, BH1991b, BH1991c}. 


\section{The MRI in Ideal MHD: Simulations} \label{sec:mrisimideal}
The simulations in this section were modelled after~\cite{XXXXXXBH1991c}, hereafter called HB3. Although the techniques used in that paper were slightly different (periodic radial conditions, for example, instead of periodic plus shear), it is instructive to replicate the results with the code that will be used

\subsection{Athena 4.2: An MHD Code....}
The code in use for the simulations in this section and Chapter~\ref{chap:nonIdealXXXXXXX} and the MHD simulations of Chapter~\ref{chap:kinBragXXXX} is Athena4.2. XXXXXXXXXXXX

\subsection{Parameters}

\subsection{Results}

