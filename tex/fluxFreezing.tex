\subsection{Alfv'en's Theorem (Flux Freezing)}
In ideal MHD, the magnetic field lines thread a perfectly conducting fluid. They cannot diffuse without a finite resistivity; hence the field lines are ``frozen'' into the conducting medium. This can be seen mathematically by using vector identity~\ref{eq:nrlXXX} to write the induction equation as
\begin{equation*}
  \frac{D\vec B}{Dt}=\vec B\cdot\nabla\vec u-\vec B(\nabla\cdot\vec u)
\end{equation*}
Using the continuity equation, this becomes
\begin{equation*}
  \frac{D}{Dt}\left(\frac{\vec B}{\rho}\right)=\left(\frac{\vec B}{\rho}\right)\cdot\nabla\vec u
\end{equation*}
which shows that the magnetic field is advected in the same way as the fluid elements themselves: hence the ``frozen-in'' terminology~\cite{Ogilvie2016}. Conservation of flux can be shown by considering


XXXXXXXXX
