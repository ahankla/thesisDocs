\chapter{Astrophysics: Context and Background} \label{chap:astrophysics}
The accretion disks around supermassive black holes are often (but not always) weakly collisional, hence providing an excellent context for exploring the validity of approximating kinetic theory with a fluid closure. However, the literature on the topic of black hole accretion disks is vast, and so this chapter seeks to extract the important ideas and clarify common jargon in the field.\\
\\
The chapter will begin by motivating the selection of accretion disks (specifically, radiatively-inefficient accretion flows, RIAFs) in Section~\ref{sec:obsmotive}, which also explains some of the observational difficulties involved in seeing accretion disks. An overview of properties of these RIAFs is given in the next section (\ref{sec:propertiesAF}), followed by important mechanics of accretion disks (Section~\ref{sec:mechanics}) and a brief history of the puzzle that accretion has historically posed (Section~\ref{sec:early}). This section leads into the next chapter, whose focus will be to explain the mechanism behind accretion itself both analytically and through numerical simulations, hence introducing important concepts for Chapter~\ref{chap:kinBrag}.  

\section{Observations of Accretion Disks} \label{sec:obsmotive}
With the advent of increasingly high-resolution telescopes and interferometry networks, it is possible to probe structures on the order of the event horizons of supermassive black holes~\cite{EHT}. This is exciting for a number of reasons, the foremost related to this thesis being that the shape and properties of disks around black holes are steadily coming into focus. Direct observational evidence for models is important for understanding the disk radiation processes. An interesting application of these observations is as a test of general relativity through comparisons of the black hole ``shadow'' measurements and theoretical predictions~\cite{Psaltis2015}. However, evaluation of black hole shadow measurements requires simulating the disk around the black hole in order to understand where all the radiation is coming from; currently, GRMHD simulations are used~\cite{Psaltis2015}. As mentioned in the Introduction, such models assume the validity of a fluid closure. However, as we will see below, evidence for the low luminosity of certain black holes is strong, suggesting a weakly collisional origin and hence a kinetic model. Thus the main goal of this thesis to evaluate the validity of such simulations has applications even to tests of general relativity.\\
\\
Although theory has motivated existence


BH1998
Direct observation difficult.XX
Theoretically compelling reasons, also spectral signs. Young, Schneider, Schechtman 1981 detected only one rotation peak.


XXX explain buckets of energy.

Look at BH1998 again.

why riafs (also bh xray binaries: quiescent RIAF to cold black-body when mass accretion rate increase: Das2013), 

Moscibrodzka2014:
Sgr A*: M=4 million solar masses
radio/millimeter source: synchrotron origin, emitted by plasma
In radio: less radiation at shorter wavelengths, from disk.
Millimeter and shorter come from black hole inner region, not disk?
Average size is measured, comparable to ``shadow'': direct evidence~
lambda>1mm, Milky way electron radio signature obscures source, so might not be from Sgr A*.
Could be from disk or jets. 
hard to compare with observations because don't know electron distribution function. Can't compare GRMHD simulations! VLBI should be able to tell.
Chandra: inner AF x-rays are only 10\% of total, which rules out high-inclination of disk models.

Plant2014
Black Hole X-ray binaries: usually quiescent, have outbursts. Outbursts due to disk instabilities: emit hard X-rays in power-law (due to inverse Copmton scattering). What happens during the transition? Accretion rate increases, except see broadened Fe lines at ISCO suggest disk truncation. Use black hole Gx 339-4. XMM-Newton (European Space Agency), Suzaka (NASA x-ray satellite).

Chan2015
GRMHD can't capture low-density electron thermodynamics-> can't predict accretion flows and spectra because dominated by electrons. Use simplified emission models instead. But these all fit the spectra equally well! BUT EHT measurements rule out strong funnel emission.
Why don't know electron distribution? HEating not understood (very anisotropic), electron acceleration?
Explore lots of models, but still MHD!

Oezel2000
X-ray from bremsstrahlung. ADAF=low mass accretion: optically thin, bremsstrahlung at outer region, synchrotron at inner. Inverse compton scattering becomes important at higher mass accretion rates.
Hot flow=overall low luminosity, synchrotron peak in radio, X-ray from bremsstrahlung, Comptonization is negligible compared to bremsstrahlung. Two-temperature model, but electrons maxwellian.

Broderick2009
IF Sgr A* is RIAF, then fit orientation/magnitude of spin. Uses mm-VLBI observations.ab initio ``well beyond reach''

Broderick2011
Still have massive uncertainties (spin 0, 64-.86 1sigma 2sigma). BUT favour length-to-width of 2.4:1 (so fat, not thin). ``ab initio calculations are beyond our present capability''. Also have gravitational lensing in the horizon vicinity. Compare physical and phenomenological models...applicability of MHD unclear (Sharma 2006,2007). PROBLEM: do not model non-thermal equilibrium electrons, which is what we see! (power law)


Dexter2013
Black hole spin tilt with respect to disk means more undefined parameters, but can explain time-dependent Sgr A* luminosity. Magnetic field ad hoc assumptions.

Mccourt2016
Use gas clouds to measure accretion rate? G2 cloud is highly eccentric, might trigger Sgr A* outburst. Watch with EHT.
%------------------------------------------------------%

\section{Properties of Hot Accretion Flows} \label{sec:propertiesAF}
Accretion flows are generally characterized by their temperature, their radiative efficiency, and/or their thickness. These types and the relationships among them will be clarified and discussed soon enough: we first note briefly that accretion does not have to lead to the formation of a disk. Indeed, some of the earliest studies on accretion concerned matter falling in radially and uniformly from all directions onto a compact object: spherical accretion~\cite{Bondi?XX}. However, such accretion flows are unlikely to occur in nature because matter will almost always be rotating with respect to the compact object and hence have angular momentum. \\
\\
AFs...

Virial: half radiated, half kinetic energy~\cite{BH1998}.
Temperature estimates complicated by not knowing how radiation is vertically transported~\cite{BH1998}. 

%------------------------------------------------------%


\section{Radiation Processes}
In order to match models with observations, we need to know what kind of light will be detected. This issue requires knowing how particles are accelerated, an active subject of research~\cite{XXXXXXX}. BH1998 gives review. 

observations...


%------------------------------------------------------%


\section{Mechanics of Accretion Disks}\label{sec:mechanics}
This section presents a series of calculations pertaining to important elements of accretion disks. The issue of the mechanism behind accretion itself is actually somewhat thorny and was under discussion for a long time, so we begin by assuming that the disk already exists and look at how such a disk functions. Accretion itself will be discussed in Section~\ref{sec:history} and explained more thoroughly in Chapter~\ref{chap:compMRI}. 


\subsection{Differential Rotation} \label{ssec:difrot}
-not always good approximation (Magnetic disks a la begelman)

\subsection{Eddington Luminosity}
Radiation from a compact object generates a radiation pressure that begins to counter the gravitational force from the central object. When the luminosity is high enough, the object's gravity is no longer enough to keep it together. When the radiation pressure and gravity exactly cancel, the system is in equilibrium. We can derive the value of the luminosity at this equilibrium, the ``Eddington luminosity'' or ``Eddington limit'' as follows (following~\citet{Spruit2009}):\\
\\
Assume the radiation has a flux $F_{\textrm{rad}}$ and an opacity, or scattering cross-section times mass, of $\kappa$ (for ionized hydrogen, $\kappa=\sigma_T/m_p$ where $m_p$ is the mass of the proton and $\sigma_T$ is the Thomson scattering cross-section of the electron). Then the force balance is given by:
\begin{equation}
\nabla\Phi=\frac\kappa cF_{\textrm{rad}}
\end{equation}
The luminosity is defined as the total energy output per time, so we can get it from the flux (amount of light per area per time) by integrating over a surface, which we assume to be spherical:
\begin{align}
L&=\int_SF_{\mathrm{rad}}\cdot dS=\frac{c}\kappa \int_S \nabla\Phi\\
&=\frac{c}\kappa \int_V \nabla^2\Phi=\frac{4\pi Gc}\kappa\int_V\rho dV\\
&=\frac{4\pi GMc}\kappa
\end{align}
Note that the force balance equation does not include any forces other than pressure-gravity equilibrium and the radiation pressure; magnetic forces, for example, are excluded. \\
\\
In numbers, the Eddington luminosity is given by:
\begin{equation}
  L_E=1.3x10^{38}\frac{M}{M_\odot}~\mathrm{ergs~s^{-1}}
\end{equation}
where $M_\odot=1.989x10^{33}$~g is the mass of the sun. In comparison, the total luminosity of the sun is on the order of $10^{33}~\mathrm{ergs~s^{-1}}$~\cite{BH1998}.\\
\\
Black holes can exceed the Eddington luminosity because the energy contained in the infalling matter does not have to be radiated; it can simply be swallowed by the black hole and increase its mass. At high accretion rates, radiation could become trapped by the flow and advect into the black hole. This could lead to low observed luminosity. 

XX Bondi radius estimate XX

\subsection{Disk Scale Height}\label{ssec:scaleheight}
The scale height of a disk is a useful scale for defining distances. It comes from considering a stratified disk; that is, one that has a component of gravity in the vertical direction. As Figure~\ref{fig:stratDisk} illustrates, the gravitational force in the vertical direction is proportional to the force along the purely horizontal (``$R$'') direction and the ratio of the distances:
\begin{equation}
\frac{F_{gR}}{F_{gz}}=\frac{R}{z}
\end{equation}
Introducing the Keplerian rotation speed $\Omega^2=GM/R^3$, we have  
\begin{align*}
F_{gz}=F_{gR}\frac z R&=\frac{GM}{R^2}\frac zR\\
&=\Omega^2 z
\end{align*}
Now relate $F_{gz}$ to the disk density by considering the force balance equation for equilibrium in the vertical direction:
\begin{equation*}
-\frac1\rho\frac{dP}{dz}=F_{gz}
\end{equation*}
Assuming an isothermal equation of state $P=\rho c^2$, this equation can be integrated:
\begin{align*}
-\frac1\rho c^2\frac{d\rho}{dz}&=F_{gz}=\Omega^2z\\
\rho(z)&=\rho_0 e^{-\frac{\Omega^2}{c^2}\frac{z^2}2}=\rho_0e^{-\frac{z^2}{2H^2}}
\end{align*}
where $H^2\equiv\frac{c^2}{\Omega^2}$. $H$ can thus be interpreted as the distance a sound wave travels in a shear time (i.e., one rotation). Generally the thickness of a disk is taken to be $2H$. As will be discussed in Section~\ref{chap:compMRI,sec:XXXX}, it is often convenient to choose units such that $H=1$, resulting in numerical domains with dimensions of $(H,4H,H)$. 

\subsection{Rayleigh Stability Criterion}
In hydrodynamics, disks with a Keplerian velocity profile are stable against perturbations. To see this, simply consider perturbations about a circular orbit in an azimuthally symmetric potential $\Phi=\Phi(r,z)$. In cylindrical coordinates we have 
\begin{align}
\hat r&=\cos\theta\hat x+\sin\theta\hat y\\
\hat\theta&=-\sin\theta\hat x+\cos\theta\hat y
\end{align}
Upon taking the time derivatives, we find that 
\begin{align*}
\ddot r&=\frac{d}{dt}\dot r=\frac{d}{dt} \left(\dot r\hat r+r\frac{d\hat r}{dt}+\dot z\hat z\right)\\
&=\ddot r\hat r+2\dot r\frac{d\hat r}{dt}+r\frac{d^2\hat r}{dt^2}+\ddot{z}\hat z\\
&=\ddot r\hat r+2\dot r\dot\theta\hat\theta +r\ddot\theta\hat\theta-r\dot\theta^2\hat r+\ddot z\hat z\\
&=\left(\ddot r-r\dot\theta^2\right)\hat r+\left(\frac1r\frac{d}{dt}r^2\dot\theta\right)\hat\theta+\ddot z\hat z
\end{align*}
From Newton's laws we have
\begin{align}
F=-m\nabla\Phi=m\ddot r
\end{align}
leading to component-wise equilibrium equations:
\begin{align}
-\frac{\partial\Phi}{\partial r}&=\ddot r-r\dot\theta^2\label{eq:compwise1}\\
-\frac1r\frac{\partial\Phi}{\partial\theta}=0&=\frac1r\frac{d}{dt}\left(r^2\dot\theta\right)\label{eq:ancons}\\
-\frac{\partial\Phi}{\partial z}&=\ddot z\label{eq:compwise3}
\end{align}
Note that Eq.~\ref{eq:ancons} gives conservation of specific momentum $h_z$: i.e., $h_z=r^2\dot\theta$ is a constant. For a circular orbit, $\ddot r=0$ and we have $\frac1{r_0}\frac{\partial\Phi}{\partial r}=\dot\theta^2=\Omega_0^2$, where we define $\Omega_0^2\equiv\frac{1}{r_0}\frac{\partial\Phi}{\partial r}\vert_{r_0}$. Now perturbing this circular orbit, we have 
\begin{align}
r&=r_0+\delta r\\
\theta&=\Omega_0t+\delta\theta\\
z&=0+\delta z
\end{align}
We obtain equations of motion by linearizing Eqns.~\ref{eq:compwise1}-\ref{eq:compwise3} and using the perturbed variables:
\begin{align}
-\frac{\partial\Phi}{\partial r}&\approx -\left(\frac{\partial\Phi}{\partial r}\Big|_{r_0}+\delta r\frac{\partial^2\Phi}{\partial r^2}\Big|_{r_0}\right)=-\left(\Omega_0^2r_0+\delta r\frac{\partial^2\Phi}{\partial r^2}\Big|_{r_0}\right)\nonumber\\
&\approx\ddot{\delta r}-r_0\Omega_0^2-2\Omega_0r_0\delta\dot\theta-\delta r\Omega_0^2\label{eq:pert1}\\
h_z&\approx r_0^2\Omega+r_0^2\delta\dot\theta+2r_0\Omega_0\delta r\nonumber\\
&= r_0^2\Omega_0\label{eq:pert2}\\
-\frac{\partial\Phi}{\partial z}&\approx -\left(\delta z\frac{\partial\Phi}{\partial z}\Big|_{r_0,z_0}\right)\nonumber\\
&\approx \delta\ddot z
\end{align}
From Eq.~\ref{eq:pert2}, we find that $r_0^2\delta\dot\theta=-2r_0\Omega_0\delta r$, and thus $\delta\dot\theta=-\frac2{r_0}\Omega_0\delta r=-\frac2{r_0^3}h_z\delta r$. Using this in the radial equation~\ref{eq:pert1} yields
\begin{align}
\ddot r&=2\Omega_0r_0\left(-\frac2{r_0^3}h_z\delta r\right)+\delta r\left(\Omega_0^2-\frac{\partial^2\Phi}{\partial r^2}\Big|_{r_0}\right)\\
&=-\left(3\Omega_0^2 +\frac{\partial^2\Phi}{\partial r^2}\Big|_{r_0}\right)\delta r\\
&=-\kappa_0^2 \delta r
\end{align}
where $\kappa_0^2\equiv3\Omega_0^2+\frac{\partial^2\Phi}{\partial r^2}\rvert_{r_0}$ is the epicyclic frequency. The physical significance of this becomes clear upon writing the $\theta$-equation of motion as a coupled harmonic oscillator with the radial equation: a particle will simply execute small circular orbits about its zeroth-order circular path as it travels. This will become an important mechanism distinguishing disk flow from shear flow, as energy can go into epicycles instead of turbulence. Section~\ref{sec:SHEARvsKEPLER} illustrates this point in the context of accretion. \\
\\
The epicyclic frequency appears often in astrophysical situations, although under many different guises. For example, remember that $\Omega^2=\frac1r\frac{\partial\Phi}{\partial r}$. Then we have
\begin{align*}
\frac{d\Omega^2}{dr}&=-\frac1{r^2}\frac{\partial\Phi}{\partial r}+\frac1r\frac{\partial^2\Phi}{\partial r}
\end{align*}
such that
\begin{equation*}
\frac{\partial^2\Phi}{\partial r^2}=r\frac{d\Omega^2}{dr}+\frac1r\frac{\partial\Phi}{\partial r}=r\frac{d\Omega^2}{dr}+\Omega^2
\end{equation*}
With this the epicyclic frequency becomes
\begin{align}
\kappa^2&=3\Omega^2+\frac{\partial^2\Phi}{\partial r^2}\\
&=4\Omega^2\left(1+\frac A\Omega\right)
\end{align}
where $A\equiv \frac r{4\Omega^2}\frac{d\Omega^2}{dr}$ is the Oort A-value, also expressed $A=-\frac12 d\Omega/d\ln r$. This can be further re-written:
\begin{align*}
\kappa^2&=4\Omega^2\left(1+\frac{r}{4\Omega^2}\frac{d\Omega^2}{dr}\right)\\
&=\frac1{r^2}\left(4\Omega^2r^3+r^4\frac{d\Omega^2}{dr}\right)\\
&=\frac1{r^3}\frac{d}{dr}\left(\Omega^2r^4\right)
\end{align*}
which is how it is written in, for example,~\cite{BH1998}. For Keplerian rotation profiles, $A=-\frac34\Omega$, and $\kappa^2=\Omega^2$.\\
\\
The stability of hydrodynamical disks can be done in multiple ways. Here, an intuitive argument is first presented showing that angular momentum must decrease outward for stability. This condition is put on firmer theoretical ground in the second method, which also introduces tools that will be used later in performing linear analysis of the magnetohydrodynamic equations (see Section~\ref{sec:IDEALMHDXXXX}).  \\
\\
First, we look at a physically intuitive argument. 


Equations for disk...continuity, momentum balance (will add B fields later)
Du/Dt=d2u/dt2+du/dx dx/dt

XXX put in appendix?
\subsection{Keplerian vs. Shear Flow}
XXX put this in appendix? 

%--------------------------------------------%

XX reynolds number, magnetic prandtl, etc.? XX


%------------------------------------------------------%

\section{Early Problems in Accretion}\label{sec:early}
The above sections have not addressed how accretion actually works on a fundamental level. So what exactly is accretion? \\
\\
Accretion is the outward transport of angular momentum, which means that particles that lose angular momentum drop closer to the central accreting object (in the case of this thesis, a black hole) in accordance with differential rotation (Section~\ref{ssec:difrot}). The cause of this angular moment transport is the real issue. It seems natural to explain this slowing down via friction; in an accretion disk (or torus), the matter at different radii are not moving at the same velocity (i.e. there is a shear) and hence one might think that there is a sort of coefficient of kinetic friction between particles that slows down their movement and causes them to accrete. The idea that this ``molecular'' or ``shear'' viscosity could explain accretion rates is tempting, but in reality is not supported by simulations (XX or observations?).\\
\\
Early simulations solving the MHD equations for a disk showed accretion rates on the order of XXXXXXXXXX (CITE: XXXXXXXSTILL need some simulations showing molecular rates!!XXXXXXXXXxx), XXXXXXxalso observations!XXXXXXx; however, according to the standard values of molecular viscosity (in, for example,~\citet{Spitzer1962}, the standard values of molecular viscosity are around XXXXXXXx). This fantastic difference between theory and simulations resulted in several new ideas for explaining the transport of angular momentum and led to the formulation of one of the most well-known models for thin disks--the $\alpha$-disk.\\
\\
The seminal paper of ~\citet{SS1973} explores accretion disks in the context of a binary star system. It essentially characterizes ignorance in the accretion rate via the parameter $\alpha$, defining the tangential stress $w_{r\phi}=\alpha\rho v_s^2$, where $v_s$ is the sound speed such that $\rho v_s^2/2$ is the disk matter's thermal energy density, although definitions vary to order unity across sources~\cite{SS1973}. This formulation temporarily removed the need to explain the source of the viscosity and provides a parameter that is easy to tweak in numerical simulations. Although the original paper takes $\alpha$ as a constant for simplicity, it is generally a function of radius. The relevance of the $\alpha$ parameter is apparent even today, as it is a simple way to gain intuition in accretion problems despite its debated value~\cite{PennaEA2013}. \\
\\
Despite its intuitive usefulness, the $\alpha$ prescription offers no mechanism for the transport of angular momentum. XXXXXXXXX proposed that, while pure molecular viscosity could not explain the observed accretion rates, an ``effective'' viscosity due to eddy interaction could do the job~\cite{XXXXXX}. In other words, turbulence would generate eddies whose interactions would manifest similar to a viscosity. The problem became to find the source of the turbulence that would lead to outward angular momentum transport. This question is the subject of the next section.

\subsection{Potential Sources of Turbulence}
Supposing that an effective viscosity generated by turbulence can explain observed and simulated accretion rates, the question becomes: what causes this turbulence? \\
\\
Some, influenced by laboratory fluid mechanics, believed that the sheer property of having a high Reynolds number (the product of a characteristic velocity and length scale divided by the viscosity; huge in astrophysical flows due to the large length scales involved) satisfactorily accounted for the needed turbulence. The mechanism at hand is called ``vortex stretching'': due to vortex conservation, the stretching of vortices in a shear flow increases the circulation velocity around the vortex tube. This allows for free energy to be extracted from the shear flow~\cite{BH1998}.\\
\\
Others, however, suspected that, at least in accretion disks, the flow was fundamentally different than those shear flows explored in the aforementioned fluid mechanics labs. Indeed, as demonstrated nicely in~\cite{BH1998:SectionIII.c.1} and reproduced in Section~\ref{sec:MECHANICSkeplerianvsshearflow}, Keplerian flows are stable against perturbations (i.e. experience no turbulence) where shear flows are not (given the Rayleigh stability criterion of Section~\ref{sec:rayleightcrit} is satisfied). The difference is due to epicycles in Keplerian flows, which sink the energy that would otherwise devolve into prominent disturbances. A high Reynolds number is not enough to explain the necessary turbulence.\\
\\
It was long thought (and indeed, is still under discussion) that hydrodynamic instabilities could lead to turbulence in accretion disks. One such idea proposed by~\citet{Paczynski1978} in very flat disks was that gravitational instabilities similar to the Jeans instability could lead to internal heating which would in turn limit the development of the instability, maintaining a somewhat unstable state. While the paper raised important questions concerning heat transport, the effect of the instability is small in hot disks~\cite{Spruit2009}.\\
\\
Convective instabilities have garnered the most interest in terms of generating hydrodynamic turbulence. After all, the Schwarzschild condition...XXXXXXXx\\
\\
If the mechanism for producing turbulence were global, its effect would not be captured by a local viscosity parameter.~\cite{Spruit2009} mentions several possibilities for generating turbulence this way, including waves and shocks created by tidal forces. These effects can produce accretion at rates up to $\alpha=.01$, but only in hot disks. Global disk winds, of the type suggested by~\citet{Blandford1976}, could also transport angular momentum. These magnetically-driven winds could theoretically sweep matter around in such a way as to account for the high accretion rates without a viscosity while also helping account for AGN jets~\cite{Koenigl1989}; however, the presence of these winds in all accretion disks is debated. A more universal and fundamental explanation seems more likely.\\
\\
Magnetic fields were thought to serve an amplifying role in turbulence transport. That is, with pre-existing turbulence, magnetic fields would tangle and speed along the transportation of magnetic fields~\cite{ShakuraSunyaev1973}. It was thought that the magnetic pressure and pressure due to turbulence were distinct, and that magnetic pressure would be insignificant in disk situations~\cite{LyndenBell1969}, or would require large magnetic fields on the order of $10^7-10^8$~G to balance the gravitational pressure of infalling gas~\cite{TrevesAl1988}. The magnetic field was mainly considered to be important due to consequences of cyclotron radiation as a cooling mechanism~\cite{Shapiro1973}. It was not until the early 1990s that the full significance of magnetic fields was appreciated.\\
\\
In 1991, Balbus and Hawley~\cite{BalbusHawley1991a,BalbusHawley1991b} closed the conceptual circle by showing that turbulence resulted directly from a weak magnetic field. Pre-existing turbulence was not needed; the entire sequence of generating turbulence and transporting turbulence and angular momentum could be derived as a result of a linear instability in the MHD equations (see Section~\ref{ssec:mriDeriv}). Numerous numerical simulations have since confirmed the important role of magnetic fields in accretion processes. The next chapter will give an overview of the theory of the MRI in ideal and non-ideal MHD in preparation for the comparison between Braginskii MHD and kinetic theory in Chapter~\ref{chap:kinBrag}.


