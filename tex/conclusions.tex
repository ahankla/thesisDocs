\chapter{Conclusions}
The purpose of this thesis was to find a fluid closure that approximated the physics of a weakly collisional plasma; specifically, the mirror and firehose instabilities. We have in a sense skirted the interesting parameter space, having explored resistivities from $\num{1e-4}$ to $\num{4e-4}$ with magnetic Prandtl numbers between 4 and 150, only to find that magnetic energy decayed at higher resistivity but that higher viscosity is required to reach significant levels of the viscous stress. However, such a course-grained approach was necessary in order to reveal where the interesting regimes were in the first place, since we had initially no idea what values the transport coefficients would take. Future work will fill in the gaps, both with finer values of resistivity and larger values of magnetic Prandtl number. The ambiguity of some simulations in whether their magnetic energy decays or not will also be studied by running simulations for ten times longer. \\
\\
The model used in this thesis assumed spatially- and temporally-constant values of resistivity and viscosity, although the kinetic simulations of~\citetalias{Kunz2016} suggest a spatially variable magnetic Prandlt number and~\citetalias{Sharma2006}'s work suggests temporally changing the value of viscosity. This thesis knowingly simplified these aspects of Braginskii MHD; however, another regime that would be easy to explore is the combination of anisotropic and isotropic viscosity. The interaction of both types of viscosity would have to be carefully studied before meaningful conclusions about the viscosity coefficient's profile in a weakly collisional plasma could be drawn.\\
\\
It is somewhat ironic to note that, while one of the main motivations of this paper was to reduce the overall computation time, the numbers of cpu-hours required increases linearly with viscosity. Therefore, while magnetic Prandtl numbers of 4 are run within 32 cpu-hours, the anticipated needed value of close to 500 will take upwards of four or five thousand. Although this amount of time is still orders of magnitude lower than the millions of cpu-hours required for PIC simulations, it still requires special permission to run for two weeks or so for a single trial (at least on Princeton computing clusters). A thorough parameter scan is thus not as computationally feasible as once thought, although it is still much less than an equivalent scan with PIC simulations. \\
\\
This thesis has taken a necessary step towards effectively simulating collisionless plasmas with a more computationally-manageable model, paving the way for future studies and ultimately modelling weakly collisional plasmas, including the accretion disks around supermassive black holes.

